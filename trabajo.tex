\documentclass[12pt,a4paper]{article}

\usepackage[utf8]{inputenc}
\usepackage[spanish]{babel}
\usepackage{geometry}
\usepackage{graphicx}
\usepackage{titlesec}
\usepackage{setspace}
\usepackage{array}
\usepackage{booktabs}

\geometry{margin=2.5cm}
\setstretch{1.2}

\title{\textbf{Especificación Formal del Sistema de Gestión de Componentes y Pedidos}}
\author{ }
\date{}

\begin{document}

\maketitle
\tableofcontents
\newpage

\section{Problema de Dominio}

La empresa requiere un sistema que permita gestionar de manera integrada las operaciones relacionadas con la \textbf{compra, almacenamiento, venta y facturación} de componentes.  
Actualmente, los procesos se llevan a cabo de forma manual o mediante hojas de cálculo, lo que provoca los siguientes problemas:

\begin{itemize}
    \item Dificultad para conocer el estado actual del inventario.
    \item Errores frecuentes en los procesos de facturación y registro de pedidos.
    \item Falta de trazabilidad entre pedidos, proveedores y facturas.
    \item Duplicidad de información y pérdida de tiempo en la búsqueda de datos.
\end{itemize}

El objetivo es desarrollar un sistema de información que centralice los procesos de compra, venta y almacenamiento de componentes, permitiendo una gestión más eficiente, trazable y confiable.

\section{Cuestión u Objetivo del Sistema}

Diseñar e implementar un \textbf{Sistema de Gestión de Componentes y Pedidos} que integre los módulos de:
\begin{itemize}
    \item Proveedores
    \item Inventario
    \item Pedidos
    \item Facturación
    \item Clientes
    \item Empleados y cargos
\end{itemize}

El sistema debe permitir registrar, consultar y mantener actualizada toda la información de los actores y entidades involucradas, garantizando la coherencia de los datos y la automatización de los procesos.

\section{Requisitos Funcionales}

\subsection{Módulo de Proveedores}
\begin{enumerate}
    \item Registrar nuevos proveedores con su información de contacto.
    \item Consultar los datos de proveedores existentes.
    \item Modificar información de un proveedor.
    \item Eliminar proveedores sin dependencias activas.
\end{enumerate}

\subsection{Módulo de Componentes (Info\_Componentes)}
\begin{enumerate}
    \item Registrar nuevos componentes, con tipo, descripción, características y valor unitario de compra.
    \item Consultar y listar componentes registrados.
    \item Modificar los datos de un componente.
    \item Eliminar componentes obsoletos o fuera de catálogo.
\end{enumerate}

\subsection{Módulo de Entrada de Almacén}
\begin{enumerate}
    \item Registrar entradas de componentes provenientes de proveedores.
    \item Actualizar automáticamente el inventario al registrar una entrada.
    \item Consultar el histórico de entradas al almacén.
\end{enumerate}

\subsection{Módulo de Inventario}
\begin{enumerate}
    \item Consultar el inventario general y el stock disponible.
    \item Actualizar manualmente las existencias cuando sea necesario.
    \item Calcular el valor total del inventario.
\end{enumerate}

\subsection{Módulo de Pedidos}
\begin{enumerate}
    \item Registrar nuevos pedidos realizados por clientes.
    \item Asignar a cada pedido un empleado responsable.
    \item Consultar, modificar o eliminar pedidos.
    \item Registrar los detalles de cada pedido (componentes y cantidades solicitadas).
\end{enumerate}

\subsection{Módulo de Facturación}
\begin{enumerate}
    \item Generar facturas basadas en los pedidos aprobados.
    \item Asociar cada factura a un cliente y un empleado responsable.
    \item Registrar los detalles de la factura (componentes y cantidades vendidas).
    \item Calcular el total de la factura.
    \item Consultar, modificar y eliminar facturas.
\end{enumerate}

\subsection{Módulo de Clientes}
\begin{enumerate}
    \item Registrar, consultar, modificar y eliminar clientes.
    \item Asociar cada cliente a una ciudad.
\end{enumerate}

\subsection{Módulo de Empleados y Cargos}
\begin{enumerate}
    \item Registrar empleados con su información personal.
    \item Asignar a cada empleado un cargo y salario.
    \item Consultar, modificar o eliminar empleados y cargos.
\end{enumerate}

\section{Requisitos de Información}

El sistema debe almacenar y gestionar la siguiente información estructurada:

\begin{table}[h!]
\centering
\begin{tabular}{@{}p{4cm}p{10cm}@{}}
\toprule
\textbf{Entidad} & \textbf{Atributos Principales} \\ \midrule
Proveedores & Código\_Proveedor, Nombre, Dirección, Teléfono, Nombre\_Contacto, Código\_Ciudad \\
Info\_Componentes & Id, Código\_Comp, Tipo, Características, Descripción, Vlr\_Unitario\_Compra, Código\_Proveedor \\
Entrada\_Almacén & Id, Fecha\_Entrada, Código\_Componente, Cantidad\_Comprada, Código\_Proveedor \\
Inventario & Código\_Componente, Nombre\_Componente, Cantidad, Vlr\_Unitario\_Venta \\
Pedidos & Id\_Pedido, Fecha\_Pedido, Nit\_Cliente, Estado\_Pedido, Id\_Empleado \\
Detalle\_Pedido & Id\_Detalle, Id\_Pedido, Código\_Componente, Cantidad\_Solicitada \\
Facturas & Id\_Factura, Fecha\_Factura, Nit\_Cliente, Estado\_Factura, Id\_Empleado \\
Detalle\_Factura & Id\_Detalle, Id\_Factura, Código\_Componente, Cantidad\_Vendida \\
Clientes & Nit\_Cliente, Nombre, Apellidos, Código\_Ciudad, Dirección, Teléfono \\
Ciudades & Código\_Ciudad, Nombre, Departamento \\
Empleados & Id\_Empleado, Nombre, Apellidos, Dirección, Teléfono, Código\_Cargo \\
Cargos & Código\_Cargo, Nombre\_Cargo, Salario \\ \bottomrule
\end{tabular}
\end{table}

\section{Reglas del Negocio}

\begin{enumerate}
    \item Un proveedor puede proveer varios componentes, pero cada componente pertenece a un solo proveedor.
    \item Un pedido puede incluir varios componentes.
    \item Cada pedido está asociado a un cliente y a un empleado.
    \item Cada factura se genera a partir de un pedido y puede contener varios componentes vendidos.
    \item La cantidad vendida no puede superar el stock disponible en inventario.
    \item La eliminación de entidades depende de la inexistencia de dependencias activas (por ejemplo, no se puede eliminar un cliente con pedidos pendientes).
\end{enumerate}

\section{Conclusión}

La especificación presentada define los requerimientos funcionales, de información y las reglas del negocio necesarias para el diseño de un sistema de gestión integral que automatice los procesos de compra, venta, inventario y facturación, garantizando coherencia, trazabilidad y eficiencia en las operaciones.

\end{document}
