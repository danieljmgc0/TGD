\documentclass[12pt,a4paper]{article}
\usepackage[utf8]{inputenc}
\usepackage[spanish]{babel}
\usepackage{geometry}
\usepackage{graphicx}
\usepackage{setspace}
\usepackage{titlesec}
\usepackage{hyperref}

\geometry{margin=2.5cm}
\setstretch{1.2}

\titleformat{\section}{\Large\bfseries}{\thesection.}{0.5em}{}
\titleformat{\subsection}{\large\bfseries}{\thesubsection.}{0.5em}{}

\begin{document}

\begin{titlepage}
    \centering
    {\scshape\LARGE Universidad Politécnica de Valencia\par}
    \vspace{1cm}
    {\scshape\Large Planificaicón Proyecto de Tecnologías de Base de Datos \par}
    \vspace{1.5cm}
    {\huge\bfseries Sistema de Gestión de Inventario, Pedidos y Facturación \par}
    \vspace{2cm}
    {\Large Equipo de trabajo:\\ Daniel Jiménez García \\ Sergio Múñoz Gómez \\ Daniel Jiménez García\par}
    \vspace{0.3cm}
    \vfill
    {\large Fecha: \today\par}
\end{titlepage}

\section{Descripción general del sistema de información}

El sistema de información está diseñado para gestionar el aprovisionamiento y ventas de una empresa dedicada al manejo de componentes en un almacén. 
Este sistema permite controlar el inventario, registrar las entradas de productos provenientes de proveedores, gestionar los pedidos de los clientes, generar facturas y mantener un registro del personal involucrado en cada proceso.

\subsection*{Objetivos principales}
\begin{itemize}
    \item Controlar el stock de los componentes disponibles.
    \item Registrar y gestionar entradas de mercancías desde los proveedores.
    \item Administrar los pedidos de clientes y su relación con las facturas generadas.
    \item Mantener la trazabilidad entre proveedores, empleados, clientes y componentes.
    \item Automatizar la facturación y el seguimiento de los pedidos.
\end{itemize}

\section{Descripción de los objetos de información y su estructura}

\subsection{Componentes}
Representa cada tipo de componente disponible en el sistema.

\textbf{Atributos:}
\begin{itemize}
    \item Id\_Componente (PK)
    \item Tipo\_Componente
    \item Descripción
\end{itemize}

\textbf{Relaciones:} un componente puede estar vinculado a varios registros de Inventario y Entradas de Almacén.

\subsection{Inventario}
Representa el stock actual de componentes en el almacén.

\textbf{Atributos:}
\begin{itemize}
    \item Id\_Inventario (PK)
    \item Id\_Componente (FK)
    \item Nombre
    \item Cantidad
\end{itemize}

\textbf{Relaciones:} 
\begin{itemize}
    \item 1:N con Componentes
    \item 1:N con Pedidos
\end{itemize}

\subsection{Entrada de Almacén}
Registra las entradas de productos al almacén desde los proveedores.

\textbf{Atributos:}
\begin{itemize}
    \item Id\_Almacén (PK)
    \item Fecha\_Entrada
    \item Id\_Componente (FK)
    \item Cantidad
    \item Id\_Proveedor (FK)
\end{itemize}

\textbf{Relaciones:} 
\begin{itemize}
    \item N:1 con Proveedores
    \item N:1 con Componentes
\end{itemize}

\subsection{Proveedores}
Representa las empresas proveedoras de componentes.

\textbf{Atributos:}
\begin{itemize}
    \item Id\_Proveedor (PK)
    \item Nombre\_Proveedor
    \item Código\_Ciudad
    \item Dirección
\end{itemize}

\textbf{Relaciones:} 1:N con Entradas de Almacén.

\subsection{Pedido}
Representa las solicitudes de productos realizadas por los clientes.

\textbf{Atributos:}
\begin{itemize}
    \item Id\_Pedido (PK)
    \item Fecha\_Pedido
    \item Estado\_Pedido
    \item Id\_Empleado (FK)
    \item Id\_Inventario (FK)
\end{itemize}

\textbf{Relaciones:}
\begin{itemize}
    \item N:1 con Cliente
    \item N:1 con Empleado
    \item N:1 con Inventario
\end{itemize}

\subsection{Cliente}
Representa a los clientes que realizan pedidos.

\textbf{Atributos:}
\begin{itemize}
    \item Id\_Cliente (PK)
    \item Nombre
    \item Código\_Ciudad
    \item Dirección
    \item Teléfono
\end{itemize}

\textbf{Relaciones:} 
\begin{itemize}
    \item 1:N con Pedidos
    \item 1:N con Facturas
\end{itemize}

\subsection{Factura}
Registra la venta de productos a los clientes.

\textbf{Atributos:}
\begin{itemize}
    \item Id\_Factura (PK)
    \item Fecha\_Factura
    \item Estado\_Factura
    \item Id\_Empleado (FK)
\end{itemize}

\textbf{Relaciones:}
\begin{itemize}
    \item 1:N con Cliente
    \item N:1 con Empleado
\end{itemize}

\subsection{Empleados}
Representa al personal de la empresa.

\textbf{Atributos:}
\begin{itemize}
    \item Id\_Empleado (PK)
    \item Apellidos
    \item Nombre
    \item Dirección
    \item Teléfono
\end{itemize}

\textbf{Relaciones:}
\begin{itemize}
    \item 1:N con Pedidos
    \item 1:N con Facturas
\end{itemize}

\section{Procesos funcionales principales}

\subsection{Gestión de inventario}
\begin{itemize}
    \item Alta, modificación y eliminación de componentes.
    \item Control de existencias y actualización tras entradas o ventas.
\end{itemize}

\subsection{Gestión de proveedores y entradas}
\begin{itemize}
    \item Registro de nuevos proveedores.
    \item Registro de entradas de productos al almacén.
    \item Actualización automática del inventario tras cada entrada.
\end{itemize}

\subsection{Gestión de pedidos de clientes}
\begin{itemize}
    \item Creación y seguimiento de pedidos.
    \item Asignación de empleados responsables.
    \item Actualización del estado del pedido.
\end{itemize}

\subsection{Facturación}
\begin{itemize}
    \item Generación de facturas a partir de pedidos.
    \item Registro de la fecha, estado y empleado responsable.
    \item Asociación de la factura con el cliente correspondiente.
\end{itemize}

\subsection{Gestión de empleados}
\begin{itemize}
    \item Control del personal y sus tareas.
    \item Registro de los pedidos y facturas gestionadas.
\end{itemize}

\section{Borrador del diagrama de clases UML}

A continuación, se presenta el borrador del diagrama de clases UML basado en las entidades descritas. 
Este diagrama muestra las clases principales, sus atributos y relaciones cardinales.

\vspace{0.5cm}

\textbf{Clases principales:}
\begin{itemize}
    \item \textbf{Componente}: Id\_Componente, Tipo\_Componente, Descripción.
    \item \textbf{Inventario}: Id\_Inventario, Nombre, Cantidad.
    \item \textbf{EntradaAlmacén}: Id\_Almacén, Fecha\_Entrada, Cantidad.
    \item \textbf{Proveedor}: Id\_Proveedor, Nombre\_Proveedor, Dirección.
    \item \textbf{Pedido}: Id\_Pedido, Fecha\_Pedido, Estado\_Pedido.
    \item \textbf{Cliente}: Id\_Cliente, Nombre, Dirección, Teléfono.
    \item \textbf{Factura}: Id\_Factura, Fecha\_Factura, Estado\_Factura.
    \item \textbf{Empleado}: Id\_Empleado, Nombre, Apellidos, Teléfono.
\end{itemize}

\vspace{0.5cm}

En la imagen \ref{diagrama_clases} se puede ver el diagrama de clases inicial propuesto para este problema.

\begin{figure}
\includegraphics[scale=0.3]{diagrama_clases.png}
\label{diagrama_clases}
\caption{Digrama de clases inicial}
\end{figure} 

\section{Planifición}

Por último, hemos creado un digrama de Gantt para separar las tareas a realizar entre los miembros del equipo, que se puede ver en la imagen \ref{gantt}

\begin{figure}
\includegraphics[scale=0.3]{gantt.png}
\label{gantt}
\caption{Digrama de Gantt}
\end{figure} 

\section{Firmas}
\vspace{2cm}
\begin{center}
\begin{tabular}{ccc}
\rule{5cm}{0.4pt} & \rule{5cm}{0.4pt} & \rule{5cm}{0.4pt} \\
Daniel & Sergio & Ángela \\
\end{tabular}
\end{center}

\end{document}
